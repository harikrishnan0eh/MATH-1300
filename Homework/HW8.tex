\documentclass{article}

\usepackage{calculator}
\usepackage{calculus}

\usepackage{fancyhdr}
\usepackage{extramarks}
\usepackage{amsmath}
\usepackage{amsthm}
\usepackage{amsfonts}
\usepackage{tikz}
\usepackage[plain]{algorithm}
\usepackage{algpseudocode}
\usepackage{tikz,pgfplots,multicol}
\usepackage[font=small,labelformat=empty]{caption}
\usetikzlibrary{automata,positioning,arrows,patterns}

%
% Basic Document Settings
%

\topmargin=-0.45in
\evensidemargin=0in
\oddsidemargin=0in
\textwidth=6.5in
\textheight=9.0in
\headsep=0.25in

\linespread{1.1}

\pagestyle{fancy}
\lhead{\hmwkAuthorName}
\chead{\hmwkClass\ (\hmwkClassInstructor\ \hmwkClassTime)}
\rhead{\hmwkTitle}
\lfoot{\lastxmark}
\cfoot{\thepage}

\renewcommand\headrulewidth{0.4pt}
\renewcommand\footrulewidth{0.4pt}

\setlength\parindent{0pt}

\setcounter{secnumdepth}{0}
\newcounter{partCounter}
\newcounter{homeworkProblemCounter}
\setcounter{homeworkProblemCounter}{1}
\nobreak\extramarks{Problem \arabic{homeworkProblemCounter}}{}\nobreak{}

%
% Homework Problem Environment
%
% This environment takes an optional argument. When given, it will adjust the
% problem counter. This is useful for when the problems given for your
% assignment aren't sequential. See the last 3 problems of this template for an
% example.
%
\newenvironment{homeworkProblem}[1][-1]{
    \ifnum#1>0
        \setcounter{homeworkProblemCounter}{#1}
    \fi
    \section{Problem \arabic{homeworkProblemCounter}}
    \setcounter{partCounter}{1}
    \enterProblemHeader{homeworkProblemCounter}
}{
    \exitProblemHeader{homeworkProblemCounter}
}

%
% Homework Details
%   - Title
%   - Due date
%   - Class
%   - Section/Time
%   - Instructor
%   - Author
%

\newcommand{\hmwkTitle}{HW \#7}
\newcommand{\hmwkDueDate}{March 9, 2017}
\newcommand{\hmwkClass}{MATH 1300}
\newcommand{\hmwkClassTime}{Section 005}
\newcommand{\hmwkClassInstructor}{Professor Braden Balentine}
\newcommand{\hmwkAuthorName}{\textbf{John Keller}}

%
% Title Page
%

\title{
    \vspace{2in}
    \textmd{\textbf{\hmwkClass:\ \hmwkTitle}}\\
    \normalsize\vspace{0.1in}\small{Due\ on\ \hmwkDueDate\ at 10:00am}\\
    \vspace{0.1in}\large{\textit{\hmwkClassInstructor\ \hmwkClassTime}}
    \vspace{3in}
}

\author{\hmwkAuthorName}
\date{}

\renewcommand{\part}[1]{\textbf{\large Part \Alph{partCounter}}\stepcounter{partCounter}\\}

%
% Various Helper Commands
%

% Useful for algorithms
\newcommand{\alg}[1]{\textsc{\bfseries \footnotesize #1}}

% For derivatives
\newcommand{\deriv}[1]{\frac{\mathrm{d}}{\mathrm{d}x} (#1)}

% For partial derivatives
\newcommand{\pderiv}[2]{\frac{\partial}{\partial #1} (#2)}

% Integral dx
\newcommand{\dx}{\mathrm{d}x}

% Alias for the Solution section header
\newcommand{\solution}{\textbf{\large Solution}}

% Probability commands: Expectation, Variance, Covariance, Bias
\newcommand{\E}{\mathrm{E}}
\newcommand{\Var}{\mathrm{Var}}
\newcommand{\Cov}{\mathrm{Cov}}
\newcommand{\Bias}{\mathrm{Bias}}


\usepackage{xcolor}
    \colorlet{Curve}{red!75!black}
    \colorlet{Tangent}{blue!75!black}
\usepackage{pgfplots}
    \pgfplotsset{compat=1.10}
    \usetikzlibrary{
        calc,
        intersections,
        math,
    }
    \makeatletter
        \def\parsenode[#1]#2\pgf@nil{%
            \tikzset{label node/.style={#1}}
            \def\nodetext{#2}
        }
        \tikzset{
            % define style for the points
            Point/.style={
                shape=circle,
                inner sep=0pt,
                minimum size=3pt,
            },
            add node at x/.style 2 args={
                name path global=plot line,
                /pgfplots/execute at end plot visualization/.append={
                        \begingroup
                        \@ifnextchar[{\parsenode}{\parsenode[]}#2\pgf@nil
                    \path [name path global = position line #1-1]
                        ({axis cs:#1,0}|-{rel axis cs:0,0}) --
                        ({axis cs:#1,0}|-{rel axis cs:0,1});
                    \path [xshift=1pt, name path global = position line #1-2]
                        ({axis cs:#1,0}|-{rel axis cs:0,0}) --
                        ({axis cs:#1,0}|-{rel axis cs:0,1});
                    \path [
                        name intersections={
                            of={plot line and position line #1-1},
                            name=left intersection
                        },
                        name intersections={
                            of={plot line and position line #1-2},
                            name=right intersection
                        },
                        label node/.append style={pos=1}
                    ] (left intersection-1) -- (right intersection-1)
                        node [label node]{\nodetext};
                    % ---------------------------------------------------------
                    % draw the tangent line from a bit right of the point on
                    % the curve to the intersection with the ordinate
                    % and draw the corresponding points
                    \draw [dashed] let
                        \p1=($ (left intersection-1) - (right intersection-1) $),
                        \p2=($ (left intersection-1)!sign(#1)*10mm!(right intersection-1) $),
                        %\p2=($ (left intersection+1) - (right intersection+1) $),
                        \p3=($ ({axis cs:0,0}) - (\p2) $),
                        \n1={\x3/\x1}	% slope of tangent line
                    in
                        (\p2) -- +($ {\n1}*(\x1,\y1) $)
	                        
%                        		node[right,node font=\scriptsize,gray] {$y=$\y1/\x1*sign(#1) $x+$}
%                            node [Point,fill=Tangent] (origin intersection) {}
                            node [Point,fill=Curve] at (left intersection-1) {}
                    ;
                    % ----------
                    % draw the horizontal line at the curve intersection point
                    % plus the label above/below the line
%                    \tikzmath{
%                        coordinate \c1;
%                        \c1=(left intersection-1) - (right intersection-1);
%                        \slope=\cy1/\cx1*sign(#1);
%                        \plusoffset = (#1+1);
%                    }
%                    \pgfmathsetmacro{\AboveBelow}{ \slope>0 ? "above" : "below" }
%                    \draw [dashed]
%                        ([xshift=sign(#1)*2.5mm] left intersection-1) --
%                        (left intersection-1) --
%                            node [\AboveBelow,node font=\scriptsize] {$y=\slope x+\plusoffset$}
%                        (left intersection-1 -| origin intersection) --
%                        +($ sign(#1)*(-2.5mm,0) $)
%                            coordinate [pos=0.5] (a)
%                    ;
%                    % draw the horizontal line at the ordinate intersection point
%                    \draw [dotted] (origin intersection)
%                        +($ sign(#1)*(-2.5mm,0) $) --
%                        (origin intersection);
%                    % draw vertical line left/right of the ordinate
%                    \pgfmathsetmacro{\LeftRight}{ #1<0 ? "right" : "left" }
%                    \draw [stealth-stealth] (origin intersection)
%                        +($ sign(#1)*(-1.25mm,0) $) -- (a)
%                            node [midway,\LeftRight,node font=\scriptsize] {$p$}
%                    ;
%                    % ---------------------------------------------------------
                        \endgroup
                },
            },
        }
    \makeatother
\makeatletter
\def\mathcolor#1#{\@mathcolor{#1}}
\def\@mathcolor#1#2#3{%
  \protect\leavevmode
  \begingroup
    \color#1{#2}#3%
  \endgroup
}
\makeatother


\begin{document}

\maketitle

\pagebreak

\section{Additional Problems}

\begin{enumerate}
	\item Determine where $f(x)=\text{arctan}(x^2-2x)$ is increasing.
	$$\begin{align}
		\frac{1}{1+(x^2-2x)}(2x-2) &> 0\\
		\frac{2}{x-1} &> 0\\
	\end{align}$$\\
	$$(3,\infty)$$
	\item The graph of $f(x)$ is shown and the table gives values of $(x)$ and $g'(x)$.
	\begin{center}
		\begin{minipage}[l]{0.45\linewidth}
			\vspace{0pt}
			\begin{center}
		\pgfplotsset{width=4.5cm,height=4.5cm, ymax=5.5,xmax=5.5, axis equal,xtick={1,2,3,4},yticklabels={ , , ,4},ytick={1,2,3,4}}
%		\pgfplotsset{xmin=-10,xmax=10,ymin=-4,ymax=6,soldot/.style={color=blue,only marks,mark=*}
			\begin{tikzpicture}
			\begin{axis}[axis lines=middle,minor y tick num=0,minor x tick num=0,tick style={black},grid style={solid, gray!20}]
				\addplot[domain=0:2] {2*x};
				\addplot[domain=2:4] {-2*x+8};
				\addplot+[only marks,mark=*,mark options={fill=black,scale=0.7},text mark as node=false,black] coordinates {(0,0)(2,4)(4,0)};
			\end{axis}
		\end{tikzpicture}
		\end{center}
		\end{minipage}
		\begin{minipage}[l]{0.45\linewidth}
			\vspace{0pt}
				\begin{tabular}{c|cccc}
		        $x$     & 0 & 1 & 2 & 3   \\ \hline
		        $g(x)$  & 4 & 3 & 2 & 1 \\ \hline
		        $g'(x)$ & -1.1 & -0.9 & -1.2 & -0.8  
				\end{tabular}
		\end{minipage}
	\end{center}
	\begin{center}
		(The function $f(x)$ is piecewise linear)
	\end{center}
		\begin{enumerate}
			\item Given $h(x)=f(g(x))$, find $h'(1)$.\newline
			-1.8
			\item Given $k(x)=g(f(x))$, find $k'(3)$.\newline
			1.44
			\item Given $l(x)=g(g(x))$, find $l'(x)$.\newline
			$g'(g(x))\cdot g'(x)$
			\item Given $m(x)=\sqrt{f(x)}$, find $m'(1)$.\newline
			$\frac{1}{\sqrt{2}}$
		\end{enumerate}
	\item The length of the day in Boulder (Latitude 40 N) can be modeled approximately by $$l(t)=-3\cos\Big(\frac{2\pi}{365}(t+10)\Big)+12$$ where $l$ is given in hours and $t$ is the day of the year.
	\begin{enumerate}
		\item Evaluate $l(355)$; fully interpret the result in the context of this problem, including units.
		$$\begin{align}
			l(355)&=-3\cos\Big(\frac{2\pi}{365}(355+10)\Big)+12\\
			&=9 \text{ hours}
		\end{align}$$
		This answer simply means that on day number 355, the length of the day will be approximately 9 hours long.
		\item Evaluate $l'(265)$; fully interpret the result in the context of this problem, including units.
		$$\begin{align}
			l'(t)&=\frac{6}{365}\pi \sin\Big(\frac{2}{365}\pi (265+10)\Big)\\
			&=-\frac{6}{365}\pi \cos(\frac{\pi}{146})\\
			&\approx -0.05 \text { hours/day}
		\end{align}$$
		This answer means that on day 265, the length of the day is changing at a rate of -0.05 hours per day.
		\item Calculate when $l'(t)$ is largest. Explain.
		\begin{center}
		\pgfplotsset{width=14cm,height=7cm,ymax=0.06,ymin=-0.06}
%		\pgfplotsset{xmin=-10,xmax=10,ymin=-4,ymax=6,soldot/.style={color=blue,only marks,mark=*}
			\begin{tikzpicture}
			\begin{axis}[axis lines=middle,minor x tick num=1,minor y tick num=1,tick style={black},grid style={solid, gray!20}]
				%\addplot[samples=200][domain=-5:500]{-3*cos(deg(((2*3.145)/(365))*(x+10)))+12};
				\addplot[samples=200][domain=-5:300]{(6/365)*3.145*sin(deg((2/(365))*3.145*(x+10)))};
				\addplot[domain=0:175,dotted]{0.052};
				%\addplot+[only marks,mark=*,mark options={fill=black,scale=0.7},text mark as node=false,black] coordinates {(0,0)(2,4)(4,0)};
			  \legend{$l'(x)$}
			\end{axis}
		\end{tikzpicture}
		\end{center}
		The largest value for $l'(t)$ is approximately 0.052 at 80 days.
	\end{enumerate}
	\item The U.S. gross domestic product can be modeled by $$P(t)=4.351e^{0.0368t}$$ where $P$ is given in billions of dollars and $t$ is years since 1970.
	 \begin{enumerate}
	 	\item Find $P(244)$; fully interpret the result in the context of this problem, including units.
	 	$$\begin{align}
	 		P(244)&=4.351e^{0.0368(244)}\\
	 		&= 34,530.8 \text{ billion dollars}
	 	\end{align}$$
	 	This answer means that in the year 1970+244, the GDP can be estimated to be \$34,530.8 billion using the model $P(t)=4.351e^{0.0368t}$.
	 	\item When was the GDP one trillion dollars?
	 	$$\begin{align}
	 		1000 &= 4.351e^{0.0368t}\\
	 		t&= 147.754 \text{ years after 1970}
	 	\end{align}$$
	 	\item How many years does it take for the GDP to double?
	 	$$\begin{align}
	 		P(0)&=4.351e^{0.0368(0)}=\$4.351b
	 		4.351 * 2 &= 4.351e^{0.0368t}\\
	 		x&= 18.8355\text{ years past 1970}
	 	\end{align}$$
	 	\item What is $P'(224)$? Again, fully interpret (including units).
	 	$$\begin{align}
	 		P'(224)&= 0.160117e^{0.0368(224)}\\
	 		&= 608.715 \text{ billion dollars}
	 	\end{align}$$
	 	This answer means that at 1970+224, the GDP is changing at \$608.715 b per year.
	 \end{enumerate}
\end{enumerate}

\section{Section 3.6}

\begin{enumerate}
\setcounter{enumi}{31}
	\item Find $y'$ if $\tan^{-1}(xy)=1+x^2y$.
	$$\begin{align}
		\arctan(xy)&=1+x^2y\\
		\frac{1}{1+(xy)^2}(y+\frac{dy}{dx}x)&=2xy+\frac{dy}{dx}x^2\\
		\frac{y+\frac{dy}{dx}x }{1+(xy)^2}&=(2xy+\frac{dy}{dx}x^2)\\
		\frac{y+\frac{dy}{dx}x }{2y+\frac{dy}{dx}x }&= x(1+(xy)^2)\\
		y+\frac{dy}{dx}x&=2xy+2x^3y^2+\frac{dy}{dx}x^2+\frac{dy}{dx}x^4y^2\\
		\frac{dy}{dx}x-\frac{dy}{dx}x^2-\frac{dy}{dx}x^4y^2&=2xy+2x^3y^2-y\\
		\frac{dy}{dx}(x-x^2-x^4y^2)&=2xy+2x^3y^2-y\\
		\frac{dy}{dx}&=\frac{2xy+2x^3y^2-y}{x-x^2-x^4y^2}
	\end{align}$$
\setcounter{enumi}{40}
	\item
		\begin{enumerate}
			\item Suppose $f$ is a one-to-one differentiable function and its inverse function $f^{-1}$ is also differentiable. Use implicit differentiation to show that $$(f^{-1})'(x)=\frac{1}{f'(f^{-1}(x))}$$ provided that the denominator is not 0.
			$$\begin{align}
				f\circ f^{-1}(x)&=x\\
				f'(f^{-1}(x))\cdot (f^{-1})'(x) &= 1\\
				(f^{-1})'(x)&=\frac{1}{f'(f^{-1}(x))}
			\end{align}$$
			\item If $f(4)=5$ and $f'(4)=\frac{2}{3}$, find $(f^{-1})'(5)$.
			$$\begin{align}
				(f^{-1})'(5)&=\frac{1}{f'(f^{-1}(5))}\\
				&= \frac{1}{f'(4)}\\
				&= \frac{3}{2}
			\end{align}$$
		\end{enumerate}
	\item
		\begin{enumerate}
			\item Show that $f(x)=2x+\cos x$ is one-to-one.
			$$\begin{align}
				f'(x)&= 2-\sin(x) \qquad \text{Because f'(x) is a constant sign, f(x) is one-to-one}
			\end{align}$$
			\item What is the value of $f^{-1}(1)$?
			$$\begin{align}
				f^{-1}(1)&=
			\end{align}$$
			\item Use the formula from Exercise 41(a) to find $(f^{-1})'(1)$.
			$$\begin{align}
				(f^{-1})'(1)&=\frac{1}{f'(f^{-1}(1))}\\
				\\ \\ \\
			\end{align}$$
		\end{enumerate}
\setcounter{enumi}{21}
	\item Find the derivative of $f(x)=x\ln(\arctan x)$
	$$\begin{align}
		f(x)&=x\ln(\arctan x)\\
		u&=\arctan(x)\\
		u'&=\frac{1}{1+x^2}\\
		f'(x)&=\ln(u)+x\frac{1}{u}u'\\
		f'(x)&=\ln(\arctan x)+x\frac{1}{\arctan x}\Big(\frac{1}{1+x^2}\Big)\\
		f'(x)&=\ln(\arctan x)+\frac{x}{(\arctan x)(1+x^2)}
	\end{align}$$
\end{enumerate}


%
\end{document}