\documentclass{article}

\usepackage{fancyhdr}
\usepackage{extramarks}
\usepackage{amsmath}
\usepackage{amsthm}
\usepackage{amsfonts}
\usepackage{tikz}
\usepackage[plain]{algorithm}
\usepackage{algpseudocode}
\usepackage{tikz,pgfplots,multicol}

\usetikzlibrary{automata,positioning}

%
% Basic Document Settings
%

\topmargin=-0.45in
\evensidemargin=0in
\oddsidemargin=0in
\textwidth=6.5in
\textheight=9.0in
\headsep=0.25in

\linespread{1.1}

\pagestyle{fancy}
\lhead{\hmwkAuthorName}
\chead{\hmwkClass\ (\hmwkClassInstructor\ \hmwkClassTime)}
\rhead{\hmwkTitle}
\lfoot{\lastxmark}
\cfoot{\thepage}

\renewcommand\headrulewidth{0.4pt}
\renewcommand\footrulewidth{0.4pt}

\setlength\parindent{0pt}

\setcounter{secnumdepth}{0}
\newcounter{partCounter}
\newcounter{homeworkProblemCounter}
\setcounter{homeworkProblemCounter}{1}
\nobreak\extramarks{Problem \arabic{homeworkProblemCounter}}{}\nobreak{}

%
% Homework Problem Environment
%
% This environment takes an optional argument. When given, it will adjust the
% problem counter. This is useful for when the problems given for your
% assignment aren't sequential. See the last 3 problems of this template for an
% example.
%
\newenvironment{homeworkProblem}[1][-1]{
    \ifnum#1>0
        \setcounter{homeworkProblemCounter}{#1}
    \fi
    \section{Problem \arabic{homeworkProblemCounter}}
    \setcounter{partCounter}{1}
    \enterProblemHeader{homeworkProblemCounter}
}{
    \exitProblemHeader{homeworkProblemCounter}
}

%
% Homework Details
%   - Title
%   - Due date
%   - Class
%   - Section/Time
%   - Instructor
%   - Author
%

\newcommand{\hmwkTitle}{Quiz 1 Rewrite}
\newcommand{\hmwkDueDate}{January 27, 2017}
\newcommand{\hmwkClass}{MATH 1300}
\newcommand{\hmwkClassTime}{Section 005}
\newcommand{\hmwkClassInstructor}{Professor Braden Balentine}
\newcommand{\hmwkAuthorName}{\textbf{John Keller}}

%
% Title Page
%

\title{
    \vspace{2in}
    \textmd{\textbf{\hmwkClass:\ \hmwkTitle}}\\
    \normalsize\vspace{0.1in}\small{Due\ on\ \hmwkDueDate\ at 10:00am}\\
    \vspace{0.1in}\large{\textit{\hmwkClassInstructor\ \hmwkClassTime}}
    \vspace{3in}
}

\author{\hmwkAuthorName}
\date{}

\renewcommand{\part}[1]{\textbf{\large Part \Alph{partCounter}}\stepcounter{partCounter}\\}

%
% Various Helper Commands
%

% Useful for algorithms
\newcommand{\alg}[1]{\textsc{\bfseries \footnotesize #1}}

% For derivatives
\newcommand{\deriv}[1]{\frac{\mathrm{d}}{\mathrm{d}x} (#1)}

% For partial derivatives
\newcommand{\pderiv}[2]{\frac{\partial}{\partial #1} (#2)}

% Integral dx
\newcommand{\dx}{\mathrm{d}x}

% Alias for the Solution section header
\newcommand{\solution}{\textbf{\large Solution}}

% Probability commands: Expectation, Variance, Covariance, Bias
\newcommand{\E}{\mathrm{E}}
\newcommand{\Var}{\mathrm{Var}}
\newcommand{\Cov}{\mathrm{Cov}}
\newcommand{\Bias}{\mathrm{Bias}}

\begin{document}

\maketitle

\pagebreak

\begin{enumerate}
\setcounter{enumi}{0}
	\item State the precise definition of the following:
	\begin{enumerate}
		\item A $function\ f$: 
	\end{enumerate}
\setcounter{enumi}{3}
	\item Consider the function $f(x)=x^2-2x+1$. Let us evaluate the difference quotient $$\frac{f(x+h)-f(x)}{h}$$
		\begin{enumerate}
		\setcounter{enumii}{1}
		\item Find $f(x+h)-f(x)$. Do this by taking your expression from (a), and subtracting the original function $f(x)$. If done correctly, every remaining term should involve an $h$. 
		\begin{align*}
			f(x+h)-f(x)&=(x+h)^2-2(x+h)+1-(x^2-2x+1) \\
			&= x^2+h^2+2xh-2x-2h+1-x^2+2x-1 \\
			&= \boxed{h^2+2xh-2h}
		\end{align*}
		\\ \textbf{Error:} My error in this problem was simply writing $x^2$ instead of $h^2$. I had all the work correct up to the last step, where I wrote $x$ instead of $h$.
		\item Find $\frac{f(x+h)-f(x)}{h}$. Do this by taking your expression from (b), dividing it by $h$, and performing any cancelations. Your answer should no longer be a fraction.
		\begin{align*}
			\frac{f(x+h)-f(x)}{h}&=\frac{h^2+2xh-2h}{h} \\
			&= \frac{h(h+2x-2)}{h} \\
			&= \boxed{h+2x-2}
		\end{align*}
		\textbf{Error:} This error is simply a continuation from the previous part (b), where I wrote $x$ instead of $h$.
		\end{enumerate}
	\item State the domain of the following functions:
		\begin{enumerate}
			\item $f(x)=\frac{x+4}{x^2-9}$ \\ $$(-\infty, -3)\cup (-3,3) \cup (3,\infty)$$ \\ \textbf{Error:} My error in this problem was a simple mistake in confusing the 4 in the numerator with the answer, so I ended up writing 4 instead of the obvious 3.
			\item $f(x)=\sqrt{2-x}$ \\ $$(-\infty,2]$$ \\ \textbf{Error:} My error in this problem was forgetting that you cannot square root 0. 
		\end{enumerate}
\setcounter{enumi}{7}
	\item If $f(x)=\sqrt{x^2-3}$ and $g(x)=\ln{2x+5}$, find the following compositions:
	\begin{enumerate}
		\item $(f\circ g)(x)$
			$= \sqrt{\ln{(2x+5)}-3}$ \\ \textbf{Error:} I simply forgot to rewrite the "$\ln$" part of the function.
		\item $(g\circ f)(x)$
			$= \ln{(2(\sqrt{x^2-3})+5)}$ \\ \textbf{Error:} I simply forgot to rewrite the "$\ln$" part of the function.
	\end{enumerate}
\setcounter{enumi}{9}
	\item True of False? If true, explain why. If false, explain why OR provide a counterexample. Some of these are intentionally tricky, so be careful!
	\begin{enumerate}
	\setcounter{enumii}{1}
		\item If $x^2=4x$, then $x=4$ is \textbf{the} solution.
		\begin{align*}
			x^2 &= 4x \\
			x^2 -4x &= 0 \\
			x(x-4) &=0 \\
			4(4-4) &= 0 \\
			0(0-4) &= 0
		\end{align*}
		\textbf{False.} \\
		\\ \textbf{Error:} As you discussed in class, I divided by $x$, and because you cannot divide by 0, the only way you could do so would be if $x \neq 0$.
		\item $\sqrt{16}=\pm 4$\\
		\textbf{False.}\\ \\
		\textbf{Error:} My error in this problem was I thought of it as solving for $x^2$, but simply the $\sqrt{16}$ is a positive 4.
		\item $\frac{2}{3}x=\frac{2x}{3}$\\
		\textbf{True.}
		$$\frac{2}{3}x=\frac{2\times x}{3}=\frac{2x}{3}$$
		\textbf{Error:} My error in this problem was not explaining my answer.
		\item $\sqrt{x^2+4}=x+2$ \\
		\textbf{False.}
		\begin{align*}
		\sqrt{x^2+4}&=x+2 \\
		x + 2 &\neq (x+2)^2 \\
		\end{align*}
		\textbf{Error:} My error in this problem was not actually explaining my thoughts.
	\end{enumerate}
\end{enumerate}



\end{document}